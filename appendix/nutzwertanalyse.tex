\label{app:nutzwertanalyse}

\subsection{Backend-Architektur}

Im Folgenden wird über eine Nutzwertanalyse die beste Backend-Architektur ausgewählt.
Die hier betrachteten Alternativen sind die stärksten Ergebnisse einer vorangegangen Recherche. 

\begin{savenotes}
\begin{table}[H]
	\centering
	\begin{tabular}{lcccc}

		\rowcolor{white!15}				
		\textbf{Eigenschaft} 			& \textbf{Gewichtung}
			& \textbf{JEE-Stack\footnote{\url{http://www.oracle.com/technetwork/java/javaee/overview/index.html}, abgerufen am 08.11.2015}}
			& \textbf{GNS-Framework\footnote{ein betriebsinternes Framework zur entwicklung von Webanwendungen}}
			& \textbf{SrpingMVC\footnote{\url{https://spring.io}, abgerufen am 08.11.2015}} \\\hline		
		
		REST-Api integriert				& 3						& 5						& 0							& 4 \\
		\pbox{4cm}{Trennung von \\ Back- und Frontend}	& 4						& 5						& 0							& 4 \\						
		Dokumentation					& 2						& 3						& 1							& 5 \\
		Testbarkeit						& 2						& 2						& 1							& 3 \\
		\pbox{4cm}{Refactoring \\(ggf. durch Dritte)}	& 3						& 3						& 4							& 2 \\
		
		\rowcolor{MidnightBlue!15}
		\textbf{Gesamt}				& \textbf{14}			& \textbf{18}			& \textbf{6}				& \textbf{18} \\\hline
		\rowcolor{white!15}				
		\textbf{Nutzwert} 				& 						& \textbf{3,86}			& \textbf{1,14} 			& \textbf{3,57} \\
											
			    
	\end{tabular}
	
	\caption{Detaillierte Nutzwertanalyse bezüglich der Backend-Architektur}
	\label{tab:nutzwertanalyse_backend}
\end{table}
\end{savenotes}


\subsection{Frontend-Architektur}

Im Folgenden wird über eine Nutzwertanalyse die beste Frontend-Architektur ausgewählt.
Die hier betrachteten Alternativen sind die stärksten Ergebnisse einer vorangegangen Recherche. 

\begin{savenotes}
\begin{table}[H]
	\centering
	\begin{tabular}{lccccc}

		\rowcolor{white!15}				
		\textbf{Eigenschaft}	& \textbf{Gewichtung}
			& \textbf{JSF\footnote{\url{http://www.oracle.com/technetwork/java/javaee/javaserverfaces-139869.html}, abgerufen am 08.11.2015}}
			& \textbf{Angular\footnote{\url{https://angularjs.org}, abgerufen am 08.11.2015}}
			& \textbf{kein Framework}
			& \textbf{React\footnote{\url{https://facebook.github.io/react/}, abgerufen am 08.11.2015}} \\\hline		
		
		Bootstrap-Aufwand		& 2						& 3				& 3					& 5							& 2 \\
		Modularität				& 4						& 4				& 3					& 1							& 5 \\						
		Erweiterbarkeit			& 5						& 3				& 4					& 0 						& 5 \\
		Dokumentation			& 2						& 3				& 5					& 4 						& 4 \\
		User-Experience			& 4						& 3				& 5					& 2 						& 5 \\
		Kompatibilität			& 2						& 4				& 2					& 5 						& 2 \\
		
		\rowcolor{MidnightBlue!15}
		\textbf{Gesamt}			& \textbf{19}			& \textbf{20}	& \textbf{22}		& \textbf{17}				& \textbf{20} \\\hline
		\rowcolor{white!15}				
		\textbf{Nutzwert} 		& 						& \textbf{3,32}	& \textbf{3,79} 	& \textbf{2,11} 			& \textbf{4,26}\\
											
			    
	\end{tabular}
	
	\caption{Detaillierte Nutzwertanalyse bezüglich der Frontend-Architektur}
	\label{tab:nutzwertanalyse_frontend}
\end{table}
\end{savenotes}