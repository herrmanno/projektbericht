\label{app:lastenehft}

\subsubsection*{Zielbestimmung}

	\paragraph*{Musskriterien}
		\begin{itemize}
			\item Benutzerverwaltung
			\begin{itemize}
				\item Benutzer anlegen
				\item Benutzerdaten editieren
				\item Benutzer löschen
			\end{itemize}
			\item Wikiverwaltung
			\begin{itemize}
				\item Wiki anlegen
				\item Wikidaten editieren
				\item Wiki löschen
				\item Handbuch aus Wiki generieren
				\item Handbuch herunterladen
			\end{itemize}
		\end{itemize}

	\paragraph*{Wunschkriterien}
		\begin{itemize}
			\item Handbuchversionen
			\begin{itemize}
				\item Handbuchhistorie eines Wikis einsehen
				\item Handbuch aus Historie löschen
				\item Älteres Handbuch herunterladen
			\end{itemize}
			\item Handbuchgenerierung
			\begin{itemize}
				\item Aktuellen Fortschritt während Generierung anzeigen
				\item Erstellung für Wikis sperren, während aktuelle Generierung läuft
			\end{itemize}
			\item Gruppenverwaltung
			\begin{itemize}
				\item Gruppen anlegen
				\item Gruppen löschen
				\item Benutzer Gruppen zuteilen
				\item Rechte Gruppen zuteilen
			\end{itemize}
		\end{itemize}

	\paragraph*{Abgrenzungskriterien}
		\begin{itemize}
			\item Single-Sign-On über GNS internes ActiveDirectory
			\item Anlegen / Konfigurieren der Laufzeitumgebung innerhalb der GNS DMZ
		\end{itemize}

\subsubsection*{Produkteinsatz}

	\paragraph*{Anwendungsbereiche}
		Technischer/administrativer Anwendungsbereich.

	\paragraph*{Zielgruppen}
		Die Zielgruppe besteht ausschließlich aus Mitarbeitern der GNS mbH.
		Vornehmlich wird das Produkt durch Angestellte der Abteilung KIS benutzt werden.

	\paragraph*{Betriebsbedingungen}
		Das Produkt wird als Webanwendung bereitgestellt. Das Produkt ist über das Internet erreichbar.
		Es wird eine Up-Time von nahezu 100\% angestrebt.


\subsubsection*{Produktübersicht}
	Das Produkt ist eine Webapplikation, welche es dem Benutzer erlaubt den Inhalt eines existierenden
	Mediawikis in eine PDF- respektive HTML-Datei zu konvertieren und diese anschließend herunterzuladen.
	Hierzu kann der Nutzer selbstständig neue Wikis indizieren, um diese für das spätere Konvertieren
	bereitzustellen. Konvertierte Mediawikis können anderen Benutzern ebenfalls zum Download bereitgestellt
	werden.