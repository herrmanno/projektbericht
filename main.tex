\documentclass[12pt]{scrartcl}

% sonderzeichen encoding
\usepackage[utf8]{inputenc}
\usepackage[T1]{fontenc}
\usepackage[ngerman]{babel}

% alternating rowcolors in tables
\usepackage[table]{xcolor}
\rowcolors{2}{blue!15}{white}

% bilder
\usepackage{graphicx}

% pdf includes
\usepackage{pdfpages}

% toc als links gerendert
\usepackage[hidelinks]{hyperref}

% images in header & footer
\usepackage{fancyhdr}
\pagestyle{fancy}
\rhead{\includegraphics[width=2cm]{images/gns}}

%listing (source code)
\usepackage{listings}
\input{include/js}

% show lof and lot in toc
\usepackage[nottoc]{tocbibind}

% appendix package
\usepackage[toc,page]{appendix}

% deutscher appendix name
\addto\captionsngerman{\let\appendixtocname\appendixname%
\let\appendixpagename\appendixname}




\titlehead{\centering\includegraphics[width=6cm]{images/gns}}
\title{
	Projektbericht: Erstellung einer Webanwendung
	zur Verbesserung des Workflows des Erstellens von Benutzerhandbüchern
}
\author{
	Oliver Herrmann \\
	Geb.: 23.09.1992 \\
	GNS mbH \\
	Tel.: ---- \\
	Email: oliver.herrman@gns.de \\
	Ausbildungsberuf: Fachinformatiker Anwendungsentwicklung \\
	\\
	GNS mbH \\
	Frohnhauser Str. 67 \\
	45127 Essen
}
\date{\today, Essen}

\begin{document}


%\maketitle
\thispagestyle{empty}

\begin{center}
	\huge \bfseries
		Projektbericht: Erstellung einer Webanwendung
		zur Verbesserung des Workflows des Erstellens von Benutzerhandbüchern \\[4cm]

	\Large	\mdseries

	\textbf{Auszubildender} \\
	Oliver Herrmann \\
	Geb.: 23.09.1992 \\
	GNS mbH \\
	Tel.: +49 201 109 - 1523 \\
	Email: \href{mailto:oliver.herrmann@gns.de}{oliver.herrmann@gns.de}  \\
	Ausbildungsberuf: Fachinformatiker Anwendungsentwicklung \\[2cm]

	\textbf{Ausbildungsbetrieb} \\
	GNS mbH \\
	Frohnhauser Str. 67 \\
	45127 Essen \\[3cm]

	\today, Essen
\end{center}


\newpage

\tableofcontents
\newpage



\section{Einleitende Worte}
\label{sec:einleitende-worte}
...
blondgelockter Knabe mit kohlrabenschwarzem Haar auf die grüne Bank
sich setzte, die gelb angestrichen war.


Alice kann es einfach nicht lassen, sie muß dem weißen Kaninchen mit
der großen Uhr folgen und landet prompt im Wunderland. Auf ihrer Reise
durch diese fröhlich bunte, aber auch sehr eigenartige Welt begegnet
sie einer gestiefelten Raupe, dem verrückten Hutmacher und ist zu Gast
bei einer nicht Geburtstags-Party. Einer hinterlistigen Tigerkatze hat
es das Mädchen schließlich zu verdanken, daß sie den Zorn der
Herz-Königin auf sich zieht. So etwas kann einem eigentlich nur im
Traum passieren.


\subsection{Test}

\begin{itemize}
  \item Alice im Wunderland
  \item Till Eulenspiegel
  \item Harry Potter
  \begin{itemize}
    \item Der Stein der Weisen
    \item Kammer des Schreckens
    \item Der Gefangene von Askaban
    \item Der Feuerkelch
    \item Der Orden des Phönix
  \end{itemize}
  \item Jim Knopf
\end{itemize}

zurück zu Abschnitt \ref{sec:einleitende-worte}

\subsection{IDE}

In Abbildung \ref{fig:eclipse-ide} ist eclipse zu sehen.

\begin{figure}[!ht]
	\centering
	\includegraphics[width=0.8\textwidth]{images/eclipse1}	
	\caption{Eclipse IDE}
	\label{fig:eclipse-ide}
\end{figure}



\subsection{test2}

\begin{enumerate}
	\item erstens
	\item zweitens
\end{enumerate}


\section{Tables}

	Siehe \hyperref[tab:test]{Beispieltabelle}

	\begin{table}
	\centering
	\begin{tabular}{c|c}

		\rowcolor{gray!15}
	    Table head & Table head \\\hline

	    Some values & Some values \\\hline
    
	    Some values & Some values \\\hline
	    
	    Some values & Some values
	    

	\end{tabular}
	\caption{Testdaten}
	\label{tab:test}
	\end{table}

\clearpage

\begin{appendices}
\renewcommand{\thesection}{\arabic{section}} %\Roman{section}

\addtocontents{toc}{\protect\setcounter{tocdepth}{1}}

	\section{First appendix}		
		\subsection{First app. subsection 1}
		Lorem ipsum...

	\newpage
	\section{Rechtschreibfehler}
		\begin{center}
			\includegraphics[page=1, width=.85\textwidth]{pdf/test.pdf}		
		\end{center}

	\newpage		
	\section{Pflichtenehft}
		\label{app:pflichtenheft}

\subsection*{Produktfunktionen}

	\begin{itemize}
		\item 
			/F10/ \\
			\textbf{Geschäftsprozess:} Login \\
			\textbf{Nachbedingung Erfolg:} Benutzer ist angemeldet \\
			\textbf{Nachbedingung Fehlschlag:} Fehlermeldung wird angezeigt \\
			\textbf{Beschreibung:}
			\begin{enumerate}
				\item Benutzer gibt Anmeldedaten ein
				\item Benutzer klickt ''Login''-Button
			\end{enumerate}
		\item 
			/F20/ \\
			\textbf{Geschäftsprozess:} Wiki anlegen \\
			\textbf{Vorbedingung:} Benutzer ist angemeldet \\
			\textbf{Nachbedingung Erfolg:} Neues Wiki ist angelegt \\
			\textbf{Nachbedingung Fehlschlag:} Fehlermeldung wird angezeigt \\
			\textbf{Beschreibung:}
			\begin{enumerate}
				\item Benutzer navigiert zum Bereich ''Wikis''
				\item Benutzer klickt ''Neu''-Button
				\item Benutzer gibt Stammdaten ein
				\item Benutzer klickt ''Speichern''-Button
			\end{enumerate}
		\item 
			/F30/ \\
			\textbf{Geschäftsprozess:} Handbuch erstellen \\
			\textbf{Vorbedingung:}
			\begin{itemize}
				\item Benutzer ist angemeldet
				\item Wiki ist angelegt
			\end{itemize}
			\textbf{Nachbedingung Erfolg:} Handbuch steht zum Download bereit \\
			\textbf{Nachbedingung Fehlschlag:} Fehlermeldung wird angezeigt \\
			\textbf{Beschreibung:}
			\begin{enumerate}
				\item Benutzer navigiert zum Bereich ''Wikis''
				\item Benutzer wählt Wiki aus
				\item Benutzer klickt ''Konvertieren''-Button
				\item Konvertierungsfortschritt wird graphisch dargestellt
			\end{enumerate}
	\end{itemize}
		
	\newpage
	\section{Quellcode (Auszüge)}
		\lstinputlisting[language=JavaScript, caption=Test.js]{code/test.js}


\end{appendices}

\clearpage
\listoffigures
\newpage

\listoftables
\newpage

\end{document}